Die Aufgabe habe ich so interpretiert, dass ein Dreieck aus drei beliebigen Strecken gebildet wird. Dabei ist irrelevant, ob das so gebildete Dreieck von einer anderen Strecke durchtrennt wird. Mit dieser Aufgabeninterpretation erhält man im ersten Beispiel die gesuchten neun Dreiecke.
Die Suche nach Dreiecken habe ich als ein geometrisches Problem formuliert:

\textbf{
    Eine Auswahl von drei Strecken aus einer Menge von mindestens drei Stecken bilden ein Dreieck, sobald sich jede der ausgewählten Strecken mit den jeweils anderen schneiden.
}

Alle Streckenauswahlen, die ein Dreieck bilden, lassen sich also finden, indem alle möglichen Auswahlen daraufhin evaluiert werden, ob sie ein Dreieck bilden.

Der hier vorgestellte Algorithmus ist also vom Typ "`Brute Force"'. Mir ist keine effizientere Lösungsstrategie eingefallen.
