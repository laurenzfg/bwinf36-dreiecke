\subsection {Datenstruktur für Strecken}
Zentral für die Lösungsidee ist eine effiziente Berechnung von Schnittpunkten zwischen den Strecken.
Am effizientesten lassen sich Schnittpunkte mit mathematischen Verfahren berechnen.
Für eine Lösung mittels mathematischer Verfahren ist es nötig, die Strecken aus zwei Punkten mit einer Geraden, die durch beide Punkte verläuft, zu repräsentieren.

Ich habe mir überlegt, dass sich zu jeder möglichen Kombination aus zwei Punkten entweder eine Gerade mit der Funktiom \(y=mx+n\) oder eine Senkrechte zur x-Achse mit der Funktionsgleichung \(x=n\) durch die beiden Punkte aufstellen lässt.
Eine Senkrechte vom Typen \(x=n\) wird benötigt, wenn Anfangs- und Endpunkt der Strecke auf der gleichen x-Koordinate liegen. Schließlich müsste eine solche lineare Funktion über eine unendliche Steigung verfügen.
Ansonsten wird eine lineare Funktion vom Typen \(y=mx+n\) aufgestellt.

Die x- und y-Koordinaten des Anfangs- und Endpunktes einer jeden Strecke werden in je zwei Aggregaten in einer "`Strecke"'-Klasse (\texttt{line}) gespeichert.
Der Konstruktor berechnet aus diesen beiden Punkten eine Geradengleichung.

Falls eine lineare Funktion verwendet wird, wird die Steigung \(m\) der Geradengleichung folgendermaßen berechnet und in der Variablen \texttt{m\_} gespeichert:

\begin{equation}
    m=\frac{y_{P2} - y_{P1}}{x_{P2} - x_{P1}}
\end{equation}

Der y-Achsenabschnitt \(n\) ergibt sich aus folgender Rechnung und wird in der Variablen \texttt{n\_} abgelegt:

\begin{equation}
    n=y_{P1} - x_{P1} \times m
\end{equation}

Falls, wie oben dargestellt, eine Senkrechte zur x-Achse benötigt wird, wird die Nullstelle dieser Senkrechten in der Variablen \texttt{n\_} vermerkt.

Um später feststellen zu können, ob es sich um lineare Funktion oder um eine Senkrechte handelt, wird in der Klasse der Bool \texttt{isLinearFunction} entsprechend gesetzt.

\subsection {Bilden der möglichen Streckenkombinationen}
Um für jede mögliche Streckenkombination das Vorhandensein eines Dreiecks prüfen zu können, müssen alle möglichen Kombination zunächst einmal aufgestellt werden. Dies habe ich mithilfe einer rekursiven Funktion implementiert.

Diese interpretiert den Streckenauswahlprozess als einen binären Baum. Jede Strecke kann entweder ausgewählt werden oder nicht.

Das bedeutet, dass die Auswahlfunktion zunächst mit einer leeren Menge als Auswahl aufgerufen wird.
Darauf kopiert sie diese Menge.
Der kopierten Menge wird die erste Strecke hinzugefügt.
In der nächsten Rekursionsstufe werden dann beide bisherigen Mengen dupliziert und jeweils die zweite Strecke hinzugefügt.

Sobald eine Menge drei Strecken enthält, kann überprüft werden, ob die enthaltenen Strecken ein Dreieck bilden. Sobald die Rekursion beendet ist, wurden alle möglichen Kombinationen überprüft.

\subsection{Streckenauswahlevaluation}
Um zu überprüfen, ob eine Menge von drei Strecken ein Dreieck bildet, muss nur überprüft werden, ob jede der drei Strecken sich mit den anderen beiden Strecken schneidet.
Dann existieren schließlich "`drei Ecken"'. Diese Überprüfung findet durch simples Ausprobieren statt.

Zwischen zwei linearen Funktion lässt sich der Schnittpunkt folgendermaßen durch Gleichsetzen ermitteln:

\begin{equation}
    \begin{aligned}
        m_1x+n_1 &= m_2x+n_2             &&\quad\vert -n_2          \\
        m_1x+n_1-n_2 &= m_2x             &&\quad\vert -m_1x         \\
        n_1-n_2 &= (m_2-m_1)x            &&\quad\vert \div(m_2-m_1) \\
        x &= \frac{n_1-n_2}{m_2-m_1}                                \\
        y &= m_1x+n_2                                               \\
    \end{aligned}
    \label{eq:linearschnitt}
\end{equation}

Der Schnittpunkt zwischen einer Geraden \(y=m_1x+n_1\) und einer Senkrechten \(x=n_2\), liegt bei der y-Koordinate, die die gerade an der x-Stelle der Senkrechten annimmt. 

\begin{equation}
    \begin{aligned}
        x &= n_2                    \\
        y &= m_1 \times n_2 + n_1   \\
    \end{aligned}
    \label{eq:senkrechtschnitt}
\end{equation}

Falls beide Funktionen Senkrechten sind, können sie sich nur schneiden, wenn Start- oder Endpunkte sich überlappen.
Ansonsten müssen beide Senkrechten parallel zueinander sein.
Schließlich sind aufeinander liegende Strecken laut Aufgabenstellung ausgeschlossen.

Abschließend muss für jeden Schnittpunkt geprüft werden, ob er zwischen dem Start- und Endpunkt beider Strecken liegt.
Schließlich können die Geraden sich außerhalb der Strecken, die nur ein Abschnitt der Geraden sind, schneiden.
Im Sinne der Aufgabenstellung zählen nur Schnittpunkte der Strecken.

Obengenannte Rechnungen sind in der Klasse der Geraden implementiert. Je nachdem, welche Strecken der Funktion übergeben wurden, führt sie die entsprechenden Rechnungen durch. Diese Funktion gibt ein \texttt{pair<bool, Coord} zurück.
Nur wenn der Bool auf true steht, ist im zweiten Feld des Paars ein Schnittpunkt vermerkt.

In einer weiteren Funktion, \texttt{isOnLine(..)}, wird, sodenn die Geraden sich schneiden, überprüft, ob die übergebene Koordinate zwischen oder auf den die Strecke begrenzenden Punkten liegt.

Wenn alle Strecken sich wie oben beschrieben kreuzen, bilden die übergebenen Strecken ein Dreieck. Dieses kann anschließend ausgegeben werden.

\subsection {Grafische Ausgabe}
TODO

LAUFZEIT --> TODO
